\chapter{Gestão}
\label{chap:gestao}


\section{Gestão dos Funcionários}

O sistema de gestão dos funcionários não substitui sistemas de gestão tradicionais, tendo funcionalidades próprias de Luderia.


O sistema deve suportar:
\begin{itemize}
    \item o registro de funcionários;
    \item a alocação de funcionários a funções, possibilitando mais de uma função;
    \item a alocação de horário de trabalho para os funcionários, indicando a função;
    \item o registro de horas extras, via o trabalho de monitoria ou o trabalho normal, e
    \item gestão de reclamações sobre os funcionários.
\end{itemize}

\section{Gestão dos jogos de aluguel}

O sistema deve suportar:
\begin{itemize}
    \item manutenção do cadastro dos jogos;
    \item indicação do estado do jogo, incluindo registro de peças faltando, peças substituídas;
    \rafael{expansão?}
    \item indicação que um jogo é extensão do outro;
    \rafael{variante em que sentido? Tipo Ticket To Ride e TTR: Europa? É necessário isso? Talvez seja uma boa dizer se o jogo está em português ou em outro idioma. Ou se ele é "language dependent", pq vai que é algo meramente iconográfico e o monitor dá conta de explicar as regras. Se bem que o aluguel para a casa do cliente não contaria com essa "vantagem" de ter alguém para explicar o jogo.}
    \item indicação que um jogo é variante de outro;
    \rafael{No caso do BGG vc pode até dizer que o Board Game Stat faz isso. Ou seja, deve ter uma API pra isso fácil}
    \item importação de informações do Board Game Geek;
    \item armazenamento de PDFs relativos ao jogo, com identificação da utilidade;
\end{itemize}