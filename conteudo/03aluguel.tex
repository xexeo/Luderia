\chapter{Empréstimo e Aluguel de Jogos}
\label{chap:aluguel}

O negócio de emprestar jogos para jogar no local é o principal chamariz da Luderia. Ele possui uma extensa coleção e mantém um registro de uso que permite analisar se mais cópias precisam ser compradas ou repostas.

Com a pandemia, a Luderia lançou o empréstimo de jogos, que podem ser retirados pelos clientes, ou entregues a eles, para jogar em casa.

Deve ficar claro que é usado o termo empréstimo para jogar na luderia e aluguel para levar para casa.

\section{O Ingresso}

A Luderia trabalha com o modelo de negócios baseado em ingresso pago, como foi apresentado no Capítulo \ref{chap:intro}

Esse modelo permite descontos, que é utilizado hoje em alguns dias, ou bônus no consumo, funcionando como um consumo mínimo, que não é usado atualmente mas é usado em algumas promoções.


\section{O Aluguel}
\rafaelx{ Por hora?! Por dia não seria melhor?}
\rafaelx{ Será que não podemos incluir um custo de depreciação? Depois de X vezes jogado/alugado ele vai ficando mais barato?}

Atualmente, os jogos são alugados por dia. É usado um  código de cores. Assim o jogo \textit{Dooble}, que é barato, é alugado com um código verde, \textit{War} na faixa amarela,  \textit{Twilight Struggle} na faixa vermelha e o \textit{Twilight Imperium: Fourth Edition} na faixa negra. 

O cálculo feito pelos donos inclui o estado do jogo, o valor do jogo no mercado, o tempo médio de partida e até uma visão de que certos jogos são jogados por pessoas que consomem mais e a chance do jogador alugar outro jogo. 


\rafaelx{ Gloomhaven tem campanhas que podem ser mais rápidas que uma partida de TS. O ideal era falar que faixa preta é o Twilight Imperium: Fourth Edition que pode durar de 4 a 8 horas e o Gloomhaven poderia até ser um faixa vermelha pra não ter dois jogos com Twilight no nome.}

Extensões são cobradas a parte, porém em alguns jogos a Luderia já inclui a expansão no valor do aluguel. Esse é o caso do \textit{Dixit}, por exemplo, que está na faixa amarela, mas uma expansão pode ser escolhida pelos jogadores como brinde.

A Luderia deseja ter possibilidades mais amplas de dar preço ao jogo, misturando um modelo por faixas, possivelmente por cores ou outro símbolo, com um modelo por preço de aluguel individual por jogo, e ainda preços por pacotes pré comprados, e preços por dia.

Além disso, a Luderia, também pretende ter descontos para associados ou frequentadores e sistemas de fidelidade, como o décimo jogo alugado ser de graça.

Jogos que vão envelhecendo ou sendo menos alugados também vão ficando mais baratos, mudando de faixa.

\section{O Cardápio de Jogos}

Já existe um cardápio de jogos impresso, mas feito artesanalmente. Esse cardápio deve ser automatizado, por meio de um site web e também de um app, e incluir, para cada jogo:
\begin{itemize}
    \item nome;
    \item descrição;
    \item nota no BGG;
    \item nome dos monitores que sabem jogá-lo;
    \item tempo médio;
    \item número de jogadores de acordo com o jogo;
    \item número recomendável de jogadores.
\end{itemize}

No futuro outras informações podem ser necessárias.

\section{Pegando o Jogo Emprestado}

Para jogar um jogo na luderia, o cliente deve mostrar o seu cartão de consumo a um monitor, que usará o sistema para associar o jogo ao jogador, informação útil para a gestão do negócio.

\rafaelx{ Caramba! Não tinha pensado nesse estilo de aluguel não. Isso é um sistema muito puxado. Particularmente, não acho maneiro. O aluguel que tinha pensado era pra pessoa levar o jogo pra casa, tipo videolocadora. Foi isso que as luderias passaram a fazer pra sobreviver durante a pandemia. O grande negócio que uma luderia como a Ludus, GoB e Boards \& Burgers (atual LudoGrill) vende e ganha em cima é a cozinha. Ter os jogos é apenas o diferencial pra galera ficar mais tempo lá dentro consumindo.}

Quem pega emprestado não é obrigado a rearrumar as peças na caixa. Eles podem deixar tudo na mesa, ou podem colocar sem ordem na caixa. Isso é função do monitor.

\rafaelx{ Ok. Entendi o que vc fez, mas aí é melhor inverter. Nos dias mais movimentados, queremos uma rotatividade maior, então é melhor expulsar logo as pessoas de lá. Nos dias mais vazios vc pode pagar um ingresso fixo e ficar lá dentro.
Existe outra modalidade que é usada no sábado: o ingresso. Nessa modalidade o jogador entra na Luderia e joga quanto quiser, pagando um ingresso fixo. }

\subsection{Outras formas previstas no futuro}

Há interesse que no futuro possam existir várias formas de empréstimo dentro da loja, de maneira a ficar em dia com o mercado. Basicamente, a Luderia quer poder optar por modelos diferentes, conforme o movimento diário. 

Os modelos a ser suportados no futuro seriam:
\begin{itemize}
\item sem ingresso
\begin{itemize}
    \item empréstimo por jogo, por tempo, por preço único;
    \item empréstimo por jogo, por tempo, por faixa de preço;
    \item empréstimo por jogo, por tempo, com valor diferenciado por jogo;
    \item empréstimo por jogo, por tempo, com faixas e valores diferenciados para alguns jogos;
    \end{itemize}
    \item com ingresso e variações;
    \begin{itemize}
    \item ingresso simples, com empréstimo à vontade;
    \item consumação mínima, incluindo ou não o modelo de empréstimo por jogo;
    \item ingresso com desconto se a consumação for maior que certo valor, por exemplo, ingresso de R\$10  que é perdoado se consumir R\$30;
\end{itemize}
\end{itemize}

%\section{Jogos de Carta}

\rafaelx{ Faz diferença? A menos que vc pense em criar dias específicos para movimentar a luderia, como a Ludus faz. Eles tem um dia de RPG, um dia de campeonato de poker, outro de campeonato de Star Wars (não lembro o nome do jogo da Galápagos)...}


\section{Estoque de Aluguel}

O estoque de jogos de aluguel precisa ser controlado, principalmente quanto a localização e estado dos jogos.

\section{Cuidados com os jogos}

A Luderia aplica uma prática de cuidado com os jogos que inclui a possibilidade de usar \textit{sleeves} em todas as cartas, plastificar tabuleiros, imprimir blocos alternativos quando há o uso de fichas de papel, etc.

Alguns jogos são mantidos mesmo que levemente deteriorados. Outros jogos são mantidos mesmo com peças faltantes, ou são colocadas peças genéricas de substituição. Por exemplo, peões podem ser perdidos no jogo Detetive e serão substituídos por peões genéricos de plástico. A Luderia tem inclusive uma pequena reserva de peças para isso, com dados, cubos e meeples de madeira.

Os \textit{sleeves} são usados nos jogos mais caros, mas não nos jogos baratos que são facilmente repostos, como \textit{Dooble}. 
\rafaelx{ Vc acha importante dizer que os jogos terão inserts como os da Bucaneiros para ajudar na arrumação? Embora isso deixe o jogo mais pesado.}


\subsection{Fichas que se esgotam}
\rafaelx{ Gostei do Rolescreve! = )}

Alguns jogos, principalmente do tipo rola-escreve e RPG, também podem exigir a impressão de fichas alternativas. Isso pode ser feito na hora, em uma impressora laser, ou na gráfica. Um sistema futuro deve ser capaz de guardar os PDF associados a essas impressões.

\rafaelx{ Uma ideia também pros rolescreve que uso em casa é plastificar as fichas dos jogadores e usar uma caneta daquelas que podem ser apagadas com um pano/papel e álcool.}

\section{Sugestão de jogos}
O sistema futuro deve ser capaz de dar sugestões de jogos. Baseado no perfil do jogador, ou em jogos previamente jogados/alugados pelo cliente, o sistema poderá indicar novos jogos para ele jogar, ou mesmo comprar. 






