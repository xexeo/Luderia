\chapter{Recepção}
\label{chap:recepcao}

As funções da recepção são:
\begin{itemize}
    \item receber os clientes;
    \item indicar a loja, o setor de empréstimo, o setor de aluguel, restaurante e mesas, conforme o interesse do cliente, e
    \item registrar o cliente que vai para as mesas, fornecendo um cartão de consumo, e avisando do consumo mínimo, quando houver.
\end{itemize}

\section{Recepção dos clientes}

A recepção dos clientes visa deixá-los mais à vontade na Luderia. Além disso, ela procura contar todos os clientes que entram e, se possível, cadastrá-los. O recepcionista tem como função conseguir o cadastro.

Para isso é necessário que haja um terminal na recepção que permita:
\begin{enumerate}
    \item simplesmente indicar a entrada de um ou mais clientes;
    \item cadastrar um cliente, e
    \item associar um cliente a um cartão de consumo.
\end{enumerate}

\section{O Cartão de Consumo}

Ao entrar na Luderia o cliente recebe um cartão de consumo, que é obrigatório. 

Só é possível consumir nas mesas com um cartão de consumo. 
Não é possível entrar na área de mesas sem um cartão. Isso é garantido por uma roleta de acesso, em horários mais cheios controlada por um segurança, enquanto em horários mais vazios é observada pelo funcionário da recepção.

Na loja e na seção de empréstimo o cartão pode ou não ser usado. Se não for usado, o cliente que comprar ou emprestar algo vai receber um cartão de consumo para pagar suas compras. 

\section{Consumo Mínimo}

Em alguns momentos, os gerentes podem decidir, que além, ou no lugar, do ingresso haverá um consumo mínimo, ou \textit{ingresso reembonsável com compras}, por exemplo, sábado entre 17:00 e 23:00 horas. Cabe a recepção avisar desse valor, porém será dado 15 minutos de perdão, isto é, se a pessoa sair em menos de 15 minutos e seu cartão estiver vazio, não precisa pagar o consumo mínimo.

\section{Registro de entrada}

O registro de entrada é feito como em boates e prédios: identificação com nome, telefone e foto em câmera portátil. O cliente deve, pelo menos, dar um nome ou apelido para o cartão de consumo. Não serão permitidos apelidos considerados ofensivos ou preconceituosos.

\section{Funcionários}

A recepção usa pelo menos um funcionário dedicado a todo momento. Esse funcionário pode cumprir outros papéis, como emprestar jogos no balcão de empréstimo ou realizar a venda de um produto da loja.

De sexta às 15 horas até domingo às 21:00 a recepção fica com 2 funcionários. Nos outros horários, fica com 1 funcionário.

A recepção atualmente tem 4 funcionários alocados a ela, que se revesam para  trabalhar, de acordo com horários pré-estabelecidos: Rui Regente, Rita Ramos, Renata Rilleti, e Rosa Raposo. Os gerentes também costumam passar algum tempo por semana na recepção. Se necessário, um caixa ou um vendedor da loja também pode cobrir a posição. 

\rafaelx{ Vamos dizer que gerentes, caixas e garçons podem ser monitores também, ou eles serão exclusivamente monitores? Na Ludus eles tinham essa distinção. Na GoB tinham e depois misturou um pouco. Mas o gerente da GoB também explicava os jogos, embora não fosse formalmente um monitor, por exemplo.}