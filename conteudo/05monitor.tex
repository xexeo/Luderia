\chapter{Monitor}
\label{chap:monitor}

\section{Monitoria}

O funcionário normal da Luderia é o monitor. Os monitores são responsáveis por: 
\begin{itemize}
    \item sugerir jogos e conversar com os jogadores;
    \item alugar os jogos;
    \item ensinar jogadores a jogar;
    \item verificar se o jogo está bem guardado pelos jogadores, e
    \item guardar o jogo corretamente.
\end{itemize}

Com o sucesso, a Luderia tem muitos monitores. Pelo menos um por salão por horário. Alguns deles são:
\begin{itemize}
    \item Maria Madalena,
    \item Mário Motta,
    \item Marcos Moreira,
    \item Muriel Madero,
    \item Melvin Moreno,
    \item Manoel Menotti,
    \item Mirthes Minoli,
    \item Maicow Moraes.
\end{itemize}


\section{Questões levantadas}

Vamos nos preocupar com o tempo que um monitor pode se dedicar a uma mesa? Existirá algum chefe dos monitores a controlar isso? 
Normalmente há um chefe dos monitores, mas até onde sei não há um controle formal. 

Os monitores ficam o tempo que for necessário para explicar a regra do jogo. Alguns até ficam juntos durante as primeiras rodadas para ver se o grupo pegou o jogo. Mas tem jogos que acabam demandando muito de um monitor, ocupando-o durante um bom tempo. Isso pode ser um problema em dias de muito movimento. 

Será necessário ter monitores dedicados a jogos com nível de complexidade diferentes? 

\rafaelx{A menos que haja uma divisão de salões de jogos mais de galera e entrada e o terceiro piso com jogos mais pesados e lentos, porque aí os próprios jogadores podem se ajudar e tirar dúvidas caso já conheçam os jogos. Ou ter um monitor dedicado pra lidar com jogos pesados que necessitam de mais tempo e atenção.}

Pode ser importante em termos de contratação de monitores e afins e até no cadastro deles é saber quantos e quais jogos o monitor domina. Ver se há overlap disso. Saber que os jogos que são mais jogados na casa precisam ser dominados por (quase) todos os monitores e que os menos pedidos podem ser dominados por um só. 

O ideal e que todo jogo possa ser ensinado por um monitor.


\section{Mestre de Jogo}

Os monitores podem atuar como mestre de jogo em RPGs ou outros jogos que necessitem desse jogador. Também podem atuar como orientadores dedicados em jogos complicados.
Para isso deve ser feita uma reserva, pois será convidado um monitor adicional para o dia. A reserva depende de disponibilidade dos monitores.

O site da Luderia deve permitir essas reservas. 

Nesse caso, o monitor receberá como hora extra. Monitores contratados não podem trabalhar como mestres de jogos independentes, pois isso causaria problemas trabalhistas. 

Este serviço é pago à parte, preços devem ser configuráveis.

Alguns mestres de jogo são independentes. O preço e modelo de cobrança não é fixo, podendo ser específico de cada um. 

Esses  mestres poderão se cadastrar e anunciar seus serviços no site da Luderia. Nesse caso, o mestre poderá entrar de graça, se cumprir alguns requisitos (todos):
\begin{itemize}
    \item oferecer pelo menos 20 horas por mês de disponibilidade;
    \item mestrar para pelo menos dois grupos diferentes;
    \item atender a pelo menos 50\% dos pedidos de atuar como mestre nos horários disponíveis.
    \item mestrar pelo menos 12 horas por mês.
\end{itemize}

O mestre independente poderá cadastrar todos os seus jogadores sob sua conta. Se o mestre independente atingir certas metas de consumo, receberá descontos e outros incentivos, como brindes e convites.

Até atender esses critérios, o mestre não tem direito a gratuidades ou descontos.

Esses critérios podem mudar ao longo do tempo, principalmente nos valores. 

Um mestre de jogo não tem que usar obrigatoriamente  esse acordo, podendo simplesmente pagar como um cliente comum. 




\section{Professores}

Da mesma forma que trabalha com mestres independentes, a Luderia também trabalha com professores de Xadrez, Go, Mah Jong, Bridge e outros jogos. 

Nesse caso, os professores se cadastram no site, indicam seu preço e como serão pagos. A Luderia cobrará uma taxa fixa dos professores por aula. Se desejarem, eles podem usar o mesmo sistema de mestre de jogo. O motivo da taxa é porque alunos não costumam consumir em uma aula.

Alguns horários não poderão ser usados para aula, devendo o professor entrar como um cliente comum.
Professores podem sempre escolher entrar na Luderia como clientes comuns.

Não há garantia de silêncio adequado para uma aula, porque é permitido barulho nas mesas de jogos e o silêncio é obrigatório no salão, não sendo viável dar uma aula lá.
\rafaelx{Gostei disso. Uma pegada meio personal trainer. A Luderia como uma academia de jogos.} 

