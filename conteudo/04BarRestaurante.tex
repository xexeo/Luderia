\chapter{Bar e Restaurante}
\label{chap:barest}

O serviço de bar e restaurante funciona como qualquer serviço desse tipo, com possibilidade de atendimento por garçons ou auto-atendimento.

\section{Gorjeta}

Todo o consumo de bar e restaurante sofre cobrança de 10\% ao pagar a conta, mesmo com auto-atendimento.

\section{O cartão de consumo}

O modelo atual é de cartão de consumo individual. Cada pessoa ao entrar recebe um cartão de consumo de papel e os garçons anotam o que foi consumido neles. Hoje isso permite que alguns pratos sejam divididos, porque o garçom pode anotar metade do valor (indicando 1/2 batata frita, por exemplo). Os garçons levam até uma calculadora barata para poder dividir com facilidade o valor.

Isso será substituído por um modelo de plástico, com \textit{QR code}, com base de dados. Já foram comprados 1000 cartões numerados. A expectativa é que, ao receber um pedido, o garçom use a leitora de um terminal portátil e registre quem está pagando por cada pedido, ou cada grupo de pedidos. 


\section{Terminais com os garçons e atendentes}

Atendentes e garçons usarão no futuro um terminal de atendimento. Esse terminal permitirá fazer os pedidos e ler o QR code dos clientes, ou digitar os últimos 3 números (de 000 a 999). 

\section{Terminais na mesa}


Algumas mesas possuirão um terminal na mesa onde os clientes poderão fazer os pedidos por si, que serão entregues por cartão. O terminal é, na verdade, um tablet preso ao canto da mesa, que possui um cardápio, onde se lê o \textit{QR code} com a câmera. Ao contrário dos terminais dos garçons, não é possível digitar o código, para evitar o roubo de códigos.

\section{App para clientes}

Há o interesse de ter um app que o cliente baixa e que se comunica com o sistema? Ele poderia não só consultar disponibilidade do jogo como ver um vídeo de regras rápidas, um faq do jogo, fazer pedidos que cairiam na cozinha, ou até fechar a conta dele. 

Esse app poderia até "dar match" em pessoas que estão dentro da Luderia, ou que frequentam a Luderia e possuem um gosto por jogos parecidos. Porque tem vezes que a pessoa vai para Luderia e leva o cano dos amigos, ou ela está indo em dupla e quer conhecer que funcionem melhor para 4 pessoas.

\section{Operações importantes de um sistema de consumo}

Algumas operações que o sistema de atendimento deve suportar:
\begin{itemize}
    \item fazer um pedido, para uma pessoa;
    \item fazer um pedido e dividir entre várias pessoas; 
    \item cancelar um pedido;
    \item anotar o motivo de cancelamento, com uma lista padrão (demorou demais, veio mal feito, pedido errado) e outros, e
    \item anotar uma observação em um item do pedido.
\end{itemize}

Não há controle de mesa, como valor da conta da mesa, porém a cada pedido deve ser entrado o número da mesa onde está o jogador. No segundo pedido, o número da mesa será colocado automaticamente, mas poderá ser trocado.

\section{Pagamento}

No momento de pagamento, caso vários cartões sejam pagos juntos, se houver cobrança de ingresso ou consumo mínimo, a conta é feita para o grupo sendo pago. Por exemplo, se houver um consumo mínimo de R\$10 por pessoa e são 5 pessoas, o consumo mínimo é de R\$50 reais para o grupo. Mesmo que um cartão tenha gasto R\$50 e o resto tenha gasto zero, estará cumprida a exigência de consumo mínimo. \rafaelx{ Gostei. Queria que a Gob tivesse isso.}

\section{Visão do futuro}

A Luderia espera ter, no futuro, um cardápio digital. O cliente, na mesa, poderá navegar no cardápio digital e incluir itens em seu pedido. Para cada item poderá fazer uma anotação em texto (sem cebola, por exemplo). Alguns itens podem ter opções, como um hambúrguer pode ter opções que se excluem, como do tipo de proteína (bovina, suína, frango ou vegetariana), ou itens a colocar (opções múltiplas), como caixas para escolher se terá tomate, alface, cebola...

No cardápio também deverá haver a opção de chamar um garçom ou monitor (ver os Capítulos \ref{chap:aluguel} e \ref{chap:monitor} para saber mais sobre os monitores).

Os pedidos ficam em um carrinho, e só são feitos quando confirmados. Os pedidos não podem ser cancelados via cardápio digital, só com os garçons.

Outra opção disponível no cardápio e mostrar a conta atual do cliente, e permitir o pagamento via cartões, ou outro modo de pagamento eletrônico, direto na mesa. Nesse caso o cartão não será ``zerado'' como é feito no caixa, mas sim indicado que já foi pago, ainda tendo que ser apresentado no caixa. É possível também entregar o cartão pago para o garçom (solicitado ao fazer o pagamento) e trocá-lo por um cartão de saída. O garçom entregará o cartão de consumo por um cartão de saída.

Para usar o cardápio digital, não é necessário fazer o login. O cliente se identificará na hora de confirmar o pedido no carrinho, por meio do cartão magnético, e indicando o número da mesa que está.

\subsection{Cardápio digital do garçon}

O garçom utiliza uma versão modificada do cardápio digital. Nela, as imagens só aparecem se solicitado em um botão, para mostrar ao cliente. Além disso, algumas funcionalidades estão habilitadas só para o garçom, que faz login no cardápio, como cancelar pedido.

Um garçom tem acesso a todos os pedidos.

\subsection{Gestão do cardápio}

Na gestão do cardápio, além de incluir e excluir itens e sub-itens (com opções múltiplas ou excludentes), deve ser também possível indicar que um item está esgotado.

Alguns itens também podem ser ligados ao estoque e marcar esgotado automaticamente.


\subsection{Cartão de saída}
Também foram comprados cartões de plástico, gravados com o logo da Luderia,  que servem de cartão de saída livre. Esses cartões são controlados pelo caixa, que os troca por cartões de consumo pago. Os cartões são numerados e só podem ser usados quando liberados pelo caixa.





